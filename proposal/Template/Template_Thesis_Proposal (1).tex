\documentclass[11pt]{article}
\usepackage[colorlinks=true,citecolor=blue]{hyperref}
\usepackage[latin1]{inputenc}
\usepackage{apacite}
\usepackage[margin=2cm,nofoot]{geometry}

\newcommand{\thetitle}{My Amazing Research Proposal}

\usepackage{fancyhdr}
\pagestyle{fancy}
\lhead{\thepage}
\rhead{\thetitle}
\cfoot{}

\begin{document}

\begin{center}
{\scshape\large \thetitle}\\[0.75\baselineskip]
Ph.D. Dissertation Proposal; Music Technology\\
Your name
\end{center}

\thispagestyle{empty}

\section*{Introduction}

In this section, you need to explain \emph{what} you want to do and  \emph{why}, \emph{in a very accessible language}\footnote{So that professors from any of the areas in the Music Research Department should be able to understand what you want to do. }.

You'll need to talk about the general research area you are addressing (context) and the limitations you are trying to overcome.

You also need to say a bit about how what you propose can improve the current field of research, although you can give more details on the Contributions section below. Also, explicitly mention here the possible impacts to audio and music technology.

Always have in mind the \emph{"So What?"} question. 

Size: \emph{half a page}.

\section*{Previous Work}

Here you describe what has been done in this direction. 

It is essential that you give credit to whoever has done something related to what you are proposing and also shortly describe the limitations of what already exists. 

Take the time to do a thorough bibliography review (\emph{not only a quick Google search!}). Although you will probably not have the space to put everything here, mention in more detail the main works related to your proposal and quickly cite less related ones.

Always try to be fair when you comment on previous works (not too positive, nor too critical). Also, be careful not to be anachronistic!

Size: \emph{a couple of paragraphs (half a page max)}
  
\section*{Proposed Research/Methodology}

In this section you will need to describe in detail how you want to achieve what you are proposing. \emph{Here you can use technical language to explain your project.} 

This is the most important section for the Music Tech Area. Remember that you need to convince the other profs. of the feasibility of what you propose.


Size: \emph{3/4 of a page}

\section*{Contributions}

This is a short paragraph on the novelty and the potential benefits of your proposal. You may already have alluded to it in the Introduction, so here cite possible applications of you research and the potential impact to the research community (knowledge), industry (products), etc.

Size: \emph{1 or 2 paragraphs}

\emph{These four sections should fit in two pages!!!}

\section*{References}

List all the works cited in the text. \emph{Don't add any works that you didn't cite!}

Size: \emph{1 page maximum}

\clearpage

\bibliographystyle{apacite}
\nocite{*}
\bibliography{proposal_2}

\end{document}
