\documentclass [12pt,letterpaper]{report}

% Standard packages
\usepackage{amsmath}		% Extra math definitions
\usepackage{graphics}		% PostScript figures
\usepackage{setspace}		% 1.5 spacing
\usepackage{longtable}          	% Tables spanning pages
\usepackage{graphicx}
\usepackage{caption}
\usepackage{subcaption}
\usepackage[ampersand]{easylist}

% Aaron's Packages
\usepackage{Style/ichago}   % The ichago references style
\usepackage{url}            % For formatting urls
\usepackage{listings}       % For formatting code
\usepackage{color}
\usepackage{sidecap}
\usepackage{wrapfig}
\usepackage{pgfplots}

\pgfplotsset{my style/.append style={axis x line=middle, axis y line=middle, xlabel={$x$}, ylabel={$y$}, axis equal }}

\definecolor{dkgreen}{rgb}{0,0.6,0}
\definecolor{gray}{rgb}{0.5,0.5,0.5}
\definecolor{mauve}{rgb}{0.58,0,0.82}

\lstset{frame=tb,
  language=C,
  aboveskip=3mm,
  belowskip=3mm,
  showstringspaces=false,
  columns=flexible,
  basicstyle={\small\ttfamily},
  numbers=none,
  numberstyle=\tiny\color{gray},
  keywordstyle=\color{blue},
  commentstyle=\color{dkgreen},
  stringstyle=\color{mauve},
  breaklines=false,
  breakatwhitespace=false,
  frame = single,
  lineskip = 0pt,
  tabsize=3
}

% Custom packages
\usepackage[first]{Style/datestamp}			% Datestamp on first page of each chapter
\usepackage[fancyhdr]{Style/McECEThesis}	% Thesis style
\usepackage{Style/McGillLogo}			% McGill University crest

% $Id: ThesisEx.tex,v 1.1 2005/06/09 12:48:46 kabal Exp $

\usepackage{color}
\def\headrulehook{\color{red}}		% Color the header rule

%===== page layout
% Define the side margins for a right-side page
\insidemargin = 1.3in
\outsidemargin = 0.8in

% Above margin is space above the header
% Below margin is space below footer
\abovemargin = 1.1in
\belowmargin = 0.75in

%========= Document start

\begin {document}

%===== Title page

\title{A Flexible Tool for the Visualization and Manipulation of Musical Mapping Networks}
\author{Aaron Henry Krajeski}
\date{\Month\ \number\year}
\organization{%
    \\[0.2in]
    \McGillCrest {!}{1in}\\	% McGill University crest
    \\[0.1in]
    Department of Music Technology\\
    Schulich School of Music, McGill University\\
    Montreal, Canada}
\note{%
    {\color{red} \hrule height 0.4ex}
    \vskip 3ex
    A thesis submitted to McGill University in partial fulfillment of the
    requirements for the degree of Master of Arts.
    \vskip 3ex
    \copyright\ \the\year\ Aaron Henry Krajeski
}

\maketitle

%===== Justification, spacing for the main text
\raggedbottom
\onehalfspacing
\pagenumbering{roman}

%===== Abstract, Sommaire & Acknowledgments
\section*{\centering Abstract}

This thesis project presents MapperGUI, a cross-platform graphical tool for the manipulation of musical mapping networks. Most digital musical instruments (DMIs) gather gestural input from musicians by way of electronic sensors and transform these data into sound through separate synthesis engines. The mapping of control inputs to synthesis parameters is arbitrary, multi-faceted and extremely important for the effectiveness of DMIs. Software tools exist to aid in this process, they attempt to render the task of musical mapping more transparent, swift and configurable. 

The libmapper software library, developed at the Input Devices and Music Interaction Laboratory, creates a standard framework for DMIs to communicate data on a distributed network and map their signals collaboratively in real-time. MapperGUI presents a graphical user interface for libmapper networks, allowing non-expert users to manipulate the text-based system. The interface aims to be flexible, such that it can accommodate the vast array of musical networks and tasks that must be performed when mapping. To this end, it provides multiple independent visualizations and interaction modes within a single framework. 

This document explores some of the issues challenging the field of musical mapping and describes the motivations behind the MapperGUI project in this context. Relevant research in the fields of data visualization and interface design is summarized and applied to the task of creating a graphical user interface for libmapper networks. Prior graphical interfaces for libmapper are examined for successful features that can be incorporated into MapperGUI. Specific implementation challenges and features of the final program are described. Insight gained from interviews with users of MapperGUI is presented, along with future work and possible extensions for the interface.

MapperGUI is available for free download as a standalone application at \url{www.libmapper.org/downloads}. All code is open-source and can be accessed at \url{https://github.com/mysteryDate/webmapper}.

\newpage

\section*{\centering R\'esum\'e}

Je suis une grande pomme de terre.

%Thesis regulations require that contributions by others in the collection of materials and data, the design and construction of apparatus, the performance of experiments, the analysis of data, and the preparation of the thesis be acknowledged.
\pagebreak

\section*{\centering Acknowledgments}

Many thanks to my thesis advisor, Professor Marcelo M. Wanderley, who directed me towards this project, offered

The work presented here would certainly not exist if not for the thousands of hours spent developing libmapper itself. 

\begin{table}
    \centering
    \begin{tabular}{p{7cm} l}
        \hline\hline
        name&role\\
        \hline
        Marcelo Wanderley& Advisor, IDMIL\\
        Joseph Malloch& libmapper, Maxmapper\\
        Stephen Sinclair& libmapper, Webmapper\\
        Vijay Ruraraju& Vizmapper\\
        Jon Wilansky& GridView, HiveView\\
        H\aa kon Knutzen, Mailis Rodrigues, Clayton Mamedes, Julie Ren\'e&Feedback\\
        Caitlin Stall-Paquet&Translation, editing, being a poulooze\\
    \end{tabular}{l l}
\end{table}

%Thesis regulations require that contributions by others in the collection of materials and data, the design and construction of apparatus, the performance of experiments, the analysis of data, and the preparation of the thesis be acknowledged.
\pagebreak


%========== Tables of contents, figures, tables
\tableofcontents
\listoffigures

\listoftables

\newpage
\chapter*{List of Acronyms}\markright{List of Terms}

\begin{longtable}{ll}
    DMI 	& 	Digital Musical Instrument\\
    GUI		& 	Graphical User Interface\\
    IDMIL &   Input Devices and Music Interaction Laboratory\\ 
    API   &   Application Programming Interface\\
    SWIG  &   Simplified Wrapper and Interface Generator\\
    OSC   &   Open Sound Control\\
    MVC   &   Model View Controller\\
    HTTP  &   HyperText Transfer Protocol\\
    HTML  &   HyperText Markup Language\\
    CSS   &   Cascading Style Sheet\\
    URL   &   Uniform Resource Locater\\
\end{longtable}

\cleardoublepage
\pagenumbering{arabic}

%%========== Chapters
%\typeout{}
%\chapter{Introduction \& Motivation}

\section{Why Develop the Use of Sonification for Affective Computing?}
In this section I will talk about why we should try to develop sonification strategies for affective display

\begin{enumerate}
\item Use when visual or verbal attention is already occupied:
	\begin{enumerate}
	\item When interacting with a visual interface
	\item When interacting with another human
	\end{enumerate}
\item Use to show bodily-based measures of affect that are not socially communicated (e.g. Electrodermal Activity)
\item Emotional Communication benefits from diversity and redundancy (best to have as many cues as possible for conveying emotion)
\item There is a need for continuous display of arousal and valence (that is, there are technologies that can produce continuous display of arousal and valence).
\item Non-speech audio seems remarkably capable of emotional communication using simple psychoacoustic cues
\end{enumerate}

\subsection{When Visual or Verbal Attention is already occupied}
\section{Affective Display}

In the literature on affective computing, most articles are written on emotion recognition or modelling.  Technologies for displaying emotional information or mediating emotional communication however are very important as they form the basis of meaningful communication with the human agent.  There are many ways of displaying emotional information.  Popular and well-researched methods use facial display, gesture or postural display, or speech prosody.  These are very effective mechanisms and highly useful.  

One disadvantage of these is that they require visual or verbal attention, so in situations where visual or verbal attention is already occupied, their use adds significantly to the cognitive load.  Another related limitation is their requirements for display.  For instance, displaying synthesized faces or gestures to communicate emotion require a visual monitor or in the case of robotics, a face or a robotic-body.  In the case of using speech for emotional communication, there is a requirement of a voice or speech for transmission.  When their is no linguistic content to be conveyed, synthesized speech is not useful.

The benefits of sound is that it can be used to monitor emotions in an eyes and speech-free fashion.  Using sound, one can use the eyes to simultaneously interact with a user interface or another human.  Similarly, one can listen to another human speaking while still being able to monitor emotions. 

\section{Why Use ``Emotional" Sounds instead of Abstract Sounds? (Sonification, Auditory Icons or Earcons?)}

In sonification, often there is no clear correspondence between data and sound a priori, so an important part of the mapping is creating a metaphor that can be quickly understood by the listener but also used to hear out subtle details so that it can be used as a data bearing medium.  The problem when using sonification for display of arousal and valence, as is being discussed presently, is that the underlying data may be more accurately presented using a mapping that \textit{doesn't} use emotional cues.  Furthermore, another mapping strategy may create a more convincing metaphor  (Give an example, maybe a hammer on a piece of wood).  So why use psychoacoustic cues determined by emotional studies?

\begin{description}
\item[Clear mappi	ng/universality] One doesn't need to explain what to listen for in the mapping.  A sound just is "happy" or "sad."
\item[``Low-Level Processing"] Just as you can listen to music in a movie and have it influence your emotions w/o having to pay attention, a sonification mapping strategy consistent w/ these cues can be analyzed on a lower level.
\item[Applies Experimental Predictions] Although there is a vast literature on auditory perception, parameter mapping sonification often applies complex auditory models for data display, each context applying different models, and in general there is no consensus on the best way to do mapping, though many ways have been proposed.  Unlike in most cases, the literature on the psychoacoustic elicitors of emotion is huge and goes back three-quarters of a century.  Further, there are now computational models that predict listeners responses to music from these low-level features (Cite Coutinho 2009).  It just makes sense to draw from a literature that is already so huge.
\item[Contributes to Emotion Research]  Developing these sorts of models provides music researchers with acoustic models that use the features they predict to convey emotion, but do not have the associations with music that music has.  Therefore, it would be reasonably impossible for a listener to ``recognize" a sonification as they could recognize a piece of music.  Therefore this sort of emotional induction would be impossible.  Same thing with Juslin's other non-acoustic mechanisms.  Abstracting into a range of just psychoacoustic features provides a level of ``cleanness" to experimental stimuli.    


\end{description}


%\section{Affective Computing}

%\section{Problem Statement}

%\section{Contributions}

%\section{Thesis Overview}



































%
\typeout{}
%!TEX root = ../thesis.tex

\chapter{Introduction \& Motivation}
%%%%%%%%%%%%
%  (start with some kind of epigraph, maybe Tufte?)
%  cite: maybe Marcelo's paper on mapping, DOT, libmapper
%	look at Joe's 1-3
%	1: separate parts
%	S. Mann, “Natural interfaces for musical expression: Physiphones and a physics-based organology,” in Proceedings of the 2007 Conference on New Interfaces for Musical Expression (NIME-07), (New York, USA), pp. 118–123, 2007.
%	2: controller any arbitrary shape
%	E. R. Miranda and M. M. Wanderley, New Digital Instruments: Control and Interaction Beyond the Keyboard, vol. 21 of The Computer Music and Digital Audio Series. Middleton, Wisconsin, USA: A-R Editions, Inc., 2006.
%	3: mapping
%	J. Ryan, “Some remarks on musical instrument design at STEIM,” Contemporary Music Review, vol. 6, no. 1, pp. 3–17, 1991.
%	probably use marcelo's paper instead
%%%%%%%%%%%%
Throughout the vast majority of human history, a musical instrument was definitively both the physical object with which the musician interacted \emph{and} the direct source of the sound created: a violin with vibrating strings, a reeded saxophone, a timpani with its membrane, etc.  With the advent of electronic sound the late 19\textsuperscript{th} century 
it became possible for interactive objects to begin to separate from the sound producing devices they control. %\tocite{Chadabe, Joel (February 2000), The Electronic Century Part I: Beginnings, Electronic Musician, pp. 74–90}
As technological development progressed, so did the capacity to divide musical instruments into independent parts. With digitization it is now not only possible to arbitrarily connect a control element to any sound synthesis dimension, but also to modify this association according to the whims of the user. Since mechanical linkages are no longer necessary in the design of musical instruments, control surfaces can, and often do, take on a variety of wild and arbitrary shapes and modes of interaction. %\tocite{somebody to support this, maybe International Conference on New Interfaces for Musical Expression. [Online]. Available: http://www.nime.org/. Accessed June 16, 2007.}
All that is necessary is for these devices to output some kind of electronic signal that other, sound producing instruments can accept. With no obvious means of implementation, the success or failure of these new digital musical instruments (DMIs) often depends on how artfully their output signals are ``mapped" to synthesis parameters.

More and more frequently, the mapping itself becomes part of the expressive element of a musical work, %\tocite{}
associating itself with both composition and performance with certain DMIs. Thus is becomes necessary for mapping to be modular and interactive: sometimes poured over in composition studios, sometimes edited mid-piece. Musicians are not necessarily computer programmers, so ideally musical mapping is something in which non-experts in DMI design could participate. This means that on top of the low-level layer of interactive mapping that is simply telling a machine to connect certain signals to others in certain ways, there needs to exist an interface to make such an activity easy, logical, intuitive and in line with the artistic process.

As the actual act of mapping is as expansive and nebulous as the instruments it hopes to assist, the design of such a mapping interface presents many interesting challenges. Due to the tremendously wide variety of possible use cases, several seemingly contradictory goals emerge: What is the best way visually represent complex musical networks while simultaneously allowing for easy manipulation of these networks? How can systems with many devices and signals be well represented while still for allowing for in-depth control of small systems? How can an interface be transparent to non-technical users while still accommodating all possible functionality that advanced users may wish to use? 

Though it may not be possible to find a perfect solution to all of the above questions, it \emph{is} feasible address each in turn and accept the best available compromise. Overall it is simply necessary to wonder: What are useful features of a graphical interface for musical mapping?

\section{Context and Motivation}

In response to the challenges of collaborative musical mapping, the libmapper protocol was created at the Input Devices and Music Interaction Laboratory (IDMIL) \shortcite{malloch}. The tool is summarized by its website as follows:

\begin{quote} 
libmapper is an open-source, cross-platform software library for declaring data signals on a shared network and enabling arbitrary connections to be made between them. libmapper creates a distributed mapping system/network, with no central points of failure, the potential for tight collaboration and easy parallelization of media synthesis. The main focus of libmapper development is to provide tools for creating and using systems for interactive control of media synthesis.\footnote{\emph{libmapper: a library for connecting things}, \url{libmapper.org} (Last accessed June, 2013)}
\end{quote}

In its most basic state, libmapper takes the form of an application programming interface (API). APIs require input in the form of text. For example, the following portion of code causes a synthesizer to announce itself and start communicating with other devices on a libmapper enabled network \shortcite{malloch}:

\begin{figure}[h!]
\begin{lstlisting}[]
#include <mapper.h>
mapper_admin_init();
my_admin = mapper_admin_new("tester", MAPPER_DEVICE_SYNTH, 8000); 
mapper_admin_input_add(my_admin, "/test/input","i")) 
mapper_admin_input_add(my_admin, "/test/another_input","f"))

// Loop until port and identifier ordinal are allocated. 
while ( !my_admin->port.locked	|| !my_admin->ordinal.locked )
{
	usleep(10000); // wait 10 ms 
	mapper_admin_poll(my_admin);
}

for (;;) 
{
	usleep(10000);
	mapper_admin_poll(my_admin); 
}
\end{lstlisting}
\caption{A sample of libmapper code}
\end{figure}

This obviously makes libmapper inaccessible to users who do not have the time or desire to read through documentation files, or those who have no experience with basic programming semantics. 
%Fig. 13. Framework for a synth-side C program using libmapper. This is the minimal code needed for a synth-side device to announce itself and communicate with other devices on the network.


%Possible use cases: Performance, composition, demonstration (of a new DMI), experimentation
%	Actuator need not necessarily to be SOUND devices, but could be vibrotactile feedback, light projectors, 

\section{Project Overview}

\section{Thesis Overview}

\section{Contributions}
%
\typeout{}
\chapter{Background}

\section{Mapping}

	Libmapper, vizmapper
	
\section{Interface Design}
\subsection{MVC}


\section{Visual Design}



%
\typeout{}
%!TEX root = ../thesis.tex

\chapter{libmapper}

The McGill Digital Orchestra project\footnote{The McGill Digital Orchestra. [Online]. Available: \url{http://www.music.mcgill.ca/musictech/DigitalOrchestra/}. Accessed July 9, 2013} began in 2006 with the aim of helping researchers  and performers in music technology to work collaboratively in creating hardware and software solutions for live performance with digital technology. The libmapper project began in response to the difficulty of creating dynamic musical mappings in a collaborative setting \shortcite{malloch}. In its most basic state, libmapper is a library for connecting things. As described by its website: 

\begin{quote} 
``libmapper is an open-source, cross-platform software library for declaring data signals on a shared network and enabling arbitrary connections to be made between them. libmapper creates a distributed mapping system/network, with no central points of failure, the potential for tight collaboration and easy parallelization of media synthesis. The main focus of libmapper development is to provide tools for creating and using systems for interactive control of media synthesis.''\footnote{libmapper: a library for connecting things. [Online]. Available: \url{libmapper.org}. Accessed June, 2013}
\end{quote}

Without libmapper, DMI designers are usually required to ``hard-code'' mappings into their designs. This has the disadvantage of being slow to modify, as it might be necessary to re-compile\footnote{A process in which human-readable code is translated into something the computer can understand. This can take anywhere from a few seconds to days.} code any time a change is made. If the DMI is built in a development environment like Max/MSP modifications can be more quickly implemented. Max/MSP is a ``high-level'' abstraction on top of machine readable code, so Max/MSP programs are prone to slowness and cross-compatibility issues, inhibiting collaboration \shortcite{jamoma}. In either implementation it is difficult for someone other than the original designer to modify mappings.

As a C\footnote{An extremely popular, multi-purpose programming language.} library, libmapper does not introduce many abstractions on top of the data and can work quickly. Any device that embeds libmapper in its code can communicate with other devices that have done the same. In a libmapper network devices communicate with one another directly, as opposed to through some centralized network device. This means that less data overall needs to be sent over the network, and failure of a single device (like the router) will not crash the entire system \shortcite{new_libmapper}, an especially dire situation during live performance.

Another advantage of libmapper, which is especially relevant to this project, is the ability to create an administrative device. These ``monitors'' can query libmapper devices for data, and thus collect data on the network overall. Monitors also are able to create, destroy and modify connections on the network. This allows for external visualization and control of a libmapper network.

%%%%%%%%%%%%%%%%%%%%%%%%%%%%%%%%%%%%%%%%%%%%%%%%%%%%%%%%%%%%%%%%%%%%%%%%%%%%%%%%%%%%%%%%%%%%%%%%%%%%%%%%%%%%%%%%%%%%%%%%%%%%%%%%%%%%%%%%%%%%%%%%%%%%%%%%%%%%%%%%%%%%%%%%%%%%%%%%%%%%%%%%%%%%%%%%%%%%%%%%%%%%%%%%%%%%%%%%%%%%%%%%%%%%%%%%%%%%%%%%%%%%%%%%%%%%%%%%%%
	\section{Open sound control and libmapper syntax} % (fold)
	\label{sec:open_sound_control_and_libmapper_syntax}

Like any communication, communication between digital devices functions well only when the devices speak the same language. In the Internet age this becomes particularly relevant: the vast array of continuously connected devices, sending and requesting information would instantly collapse if every developer coded to his or her own idiosyncrasies. To prevent this, computer scientists make use of various communication ``protocols'' when creating software. Hypertext Transfer Protocol (HTTP) is the most famous example of such a system.

At its core, libmapper builds its on language on top of the Open Sound Control (OSC) protocol, as described by \shortciteN{osc}. OSC defines the format for messages that are sent between sound producing devices (as implied by the name), but can also be used for related multimedia devices such as stage lights or vibrating motors. It provides means for flexible, high-resolution communication and was intended to replace MIDI\footnote{MIDI Manufacturers Association - The official source of information about MIDI. [Online]. Available: \url{www.midi.org}. Accessed July 11, 2013}, the 30-year-old standard for musical instrument communication. 

OSC formats messages much like Internet URLs, arbitrary strings of characters separated by `/' characters. libmapper messages also take on this format, using the structure to expose hierarchy of signals:

	\begin{itemize}
	\item\url{tstick.1/raw/accelerometer/1/x}: The data for the `x' dimension of the first accelerometer of the first instrument of class ``tstick'' on the network (see TODO for a description of the gestural controller T-Stick). Here the word ``raw'' denotes that no pre-processing has been applied to this signal. 
	\item\url{tstick.1/raw/accelerometer/2/y}: A signal transmitting the data for the same instrument as above, but the `y' dimension of the second accelerometer.
	\item\url{tstick.1/cooked/accelerometer/2/amplitude}: A ``cooked'' signal. All three dimensions of accelerometer 2 are combined to compute the overall acceleration of the point. These signals can also be cooked to expose angle and elevation as signals.
	\item\url{granul8.2/filter/evelope/frequency/low}: The data for the low-end cutoff for the shape of the filter for the instrument named ``granul8.2'' (a granular synthesizer, thus a destination device).
	\end{itemize}

This structure of signal names aims to be semantically relevant, and allows a GUI to display hierarchical structure of networks. Any one of the above signals transmits not only the signal's value, but also metadata. Signal metadata usually includes data type, length (single number vs. vector), units like volts or meters per second, maximum value and minimum value. Designers can ``tag'' signals with any extra metadata they may wish to add, such as physical position, color or owner's name. In the GUI it is necessary to allow users to view and manipulate any arbitrary kind of signal metadata.

To make signal names as coherent and consistent as possible, libmapper makes use of the \emph{Gesture Description Interchange Format} (GDIF) \shortcite{GDIF}, which provides a standard for motion capture data. Structures are given short, semantically relevant names. GDIF also provides a standard vocabulary for describing motion with dimensions such as ``weight,'' ``space,'' ``time'' and ``flow.'' Though these standards are not enforced, as libmapper signals can be given any sort of names by their creators, most extant libmapper-enabled devices use them.

%%%%%%%%%%%%%%%%%%%%%%%%%%%%%%%%%%%%%%%%%%%%%%%%%%%%%%%%%%%%%%%%%%%%%%%%%%%%%%%%%%%%%%%%%%%%%%%%%%%%%%%%%%%%%%%%%%%%%%%%%%%%%%%%%%%%%%%%%%%%%%%%%%%%%%%%%%%%%%%%%%%%%%%%%%%%%%%%%%%%%%%%%%%%%%%%%%%%%%%%%%%%%%%%%%%%%%%%%%%%%%%%%%%%%%%%%%%%%%%%%%%%%%%%%%%%%%%%%%
	% section open_sound_control_and_libmapper_syntax (end)

	\section{Structure of libmapper networks} % (fold)
	\label{sec:structure_of_libmapper_networks}

In order to maintain internal consistency, libmapper introduces a naming convention of its own. At the heart of any libmapper network are \emph{signals}, defined in \citeN{new_libmapper} as:
\begin{quote}
``Data organized into a time series. Conceptually a signal is continuous, however our use of the term signal will refer to discretized signals, without assumptions regarding sampling intervals.''
\end{quote}
Here \shortciteANP{new_libmapper} are referring to digital as opposed to analog signals (hence the use of the term ``discretized''). Notice how signals are not necessarily numeric by this definition, though they will almost certainly will be going forward. Signals are the only information actually passed from control surfaces to synthesizers, all other data structures exist to organize and label them. \emph{Source signals} is data entering libmapper from control surfaces while \emph{destination signals} belong to synthesizers and receive data. A \emph{connection} is a bridge between two signals. Once a connection is created within libmapper, a source signal begins sending its data to a destination signal. A single source signal can be connected to many destination signals (a one-to-many mapping), but at the time of the writing of this document single destination signals cannot receive input from many source signals (a many-to-one mapping). Justification for this lack of functionality is discussed in \shortciteN{new_libmapper}.

\emph{Devices} are essentially groups of signals. A device often has some kind of physical entity that makes the grouping logical, e.g. the ``T-Stick,'' which has many ``child'' signals. Within software the grouping is usually a discreet computer program. With development environments like Max/MSP users are free to group signals into devices however they may wish. As mentioned previously, libmapper devices do not send all signal data to some centralized router, and instead work directly with one another. In order to accomplish this devices must be explicitly \emph{linked}. Figure \ref{fig:libmapper_devices} demonstrates instances of libmapper devices, signals, links and connections.

\begin{figure}[ht]
\centering
	\scalebox{1}{\includegraphics{figures/libmapper_devices}}
\caption{A simple libmapper network}
\label{fig:libmapper_devices}
\end{figure}

Devices and signals can carry a variety of \emph{metadata}. Devices usually list the number of child signals they possess and their location on the network (IP address and port). As previously stated users can tag devices and signals with arbitrary metadata. Connections have a much more specific set of metadata.

%%%%%%%%%%%%%%%%%%%%%%%%%%%%%%%%%%%%%%%%%%%%%%%%%%%%%%%%%%%%%%%%%%%%%%%%%%%%%%%%%%%%%%%%%%%%%%%%%%%%%%%%%%%%%%%%%%%%%%%%%%%%%%%%%%%%%%%%%%%%%%%%%%%%%%%%%%%%%%%%%%%%%%%%%%%%%%%%%%%%%%%%%%%%%%%%%%%%%%%%%%%%%%%%%%%%%%%%%%%%%%%%%%%%%%%%%%%%%%%%%%%%%%%%%%%%%%%%%%
	% section structure_of_libmapper_networks (end)

	\section{Connection properties} % (fold)
	\label{sec:connection_properties}

Creation of links and connections is mapping from the systems perspective, but libmapper also allows for functional mapping through the modification of connections. This can be accomplished by altering certain properties possessed by every libmapper connection:

\begin{itemize}
	\item\textbf{Expression}: A mathematical equation relating the source ($x$) to destination ($y$) values. An expression of $y = x$ will simply pass through source values, while an expression of $y = 3x + 2$, will apply a linear transformation to the source data (e.g. a value of 1 will be output as 5). libmapper supports a variety of expressions, including exponential functions, trigonometric relations, comparison operators, derivation and integration. 
	\item\textbf{Range}: An array of four numbers containing the user-specified maximum and minimum values for both the source and destination signals.
	\item\textbf{Mode}: The type of connection, this influences the effect of the expression and range properties and can be one of four categories:
	\begin{itemize}
		\item \emph{Linear}: libmapper automatically scales the output such that it fits the destination range, based on the source range. For example, if a certain connection has a source range of $[0, 1]$, and a destination range of $[5, 10]$, libmapper will automatically apply an expression of $y = 5x + 5$, such that the minimum and maximum source values will correspond to the minimum and maximum destination values respectively. A source value that is outside of this source range will result in a destination value that is also outside of the range. In this mode the user cannot directly modify the expression. 
		\item \emph{Calibration}: The same as the linear mode except the source range parameter is ignored. libmapper instead polls the source signal to find the source range directly.
		\item \emph{Bypass}: Source values are sent directly through to the destination signal, as would happen with an expression $y = x$.
		\item \emph{Expression}: The user is able to set the expression to any arbitrary relation.
	\end{itemize}
	\item\textbf{Boundary}: What is to happen to data values when they extend beyond the destination range. There are four options:
	\begin{itemize}
		\item\emph{None}: Values are passed through unchanged.
		\item\emph{Clamp}: Values outside of the boundary are constrained to that value.
		\item\emph{Mute}: No values outside of the boundary are passed to the output.
		\item\emph{Wrap}: Values exceeding the maximum are ``wrapped'' back to the minimum bound and vice versa.
		\item\emph{Fold}: When the signal passes outside of the boundary, the value is inverted back onto the destination range. 
	\end{itemize}
	\item\textbf{Mute}: A true-or-false value muting and un-muting data sent over the connection.
	\item\textbf{Send as instance}: Not all signals on libmapper networks are unique and long lasting, a good example being a keypress on a keyboard. During the keypress, data like aftertouch and release can be sent, making it a bona fide signal. However, musicians constantly create and complete keypress events during performance with keyboard instruments. To maintain every keypress as a unique signal with unique metadata would be tremendously unhelpful for mapping. Also, forcing a user to map every keypress event individually would make live performance impossible.

	To support this, libmapper gives connections the \emph{send as instance} property. Sending data as an instance means that libmapper treats the connected signals as instances of a general class of signals. New instances of a signal class will be treated like previous instances and do not need to be mapped individually.
	\item\textbf{Link scope}: Not a property of connections, but of links. By default links are ``scoped'' to notify destination devices of the creation and destruction of signal instances on linked source devices. For intermediate devices, ones that function as both source and destination, this may not be the desired behavior. If device \url{A} is linked to intermediate device \url{B}, which is in turn linked to device \url{C}, then \url{C} will not be notified of instance events on \url{A} with default link scope settings. The user can modify the scope of link \url{B} $\rightarrow$ \url{C} to include \url{A} if desired.

\end{itemize}

% section connection_properties (end)

%%%%%%%%%%%%%%%%%%%%%%%%%%%%%%%%%%%%%%%%%%%%%%%%%%%%%%%%%%%%%%%%%%%%%%%%%%%%%%%%%%%%%%%%%%%%%%%%%%%%%%%%%%%%%%%%%%%%%%%%%%%%%%%%%%%%%%%%%%%%%%%%%%%%%%%%%%%%%%%%%%%%%%%%%%%%%%%%%%%%%%%%%%%%%%%%%%%%%%%%%%%%%%%%%%%%%%%%%%%%%%%%%%%%%%%%%%%%%%%%%%%%%%%%%%%%%%%%%%
	\section{libmapper bindings} % (fold)
	\label{sec:libmapper_bindings}

A final crucial libmapper feature is its multi-language \emph{bindings}. The C language is ``low-level'' in that it is very and does not allow for very abstract data structures. This makes it extremely flexible, but difficult and time consuming to use. To make libmapper more friendly for different kinds of developers, ``bindings'' have been created for the higher-level 
Python\footnote{Python Programming Language - Official Website. [Online]. Available: \url{http://www.python.org/}. Accessed July 17, 2013} 
and 
Java\footnote{java.com: Java + You. [Online]. Available: \url{java.com/en}. Accessed July 17, 2013} 
programming languages. libmapper functions are bound to other languages using the 
Simplified Wrapper and Interface Generator (SWIG)\footnote{Simplified Wrapper and Interface Generator. [Online]. Available: \url{http://www.swig.org/}. Accessed July 17, 2013}.
SWIG automatically writes a kind of dictionary that interprets function calls from other languages to the original C. Automatically generated files sit in-between the controlling code and the original library. 

Though concept of ``mapping'' itself is extremely abstract, the libmapper API places it into a concrete context. libmapper is not only means of organizing networks though the creation and destruction of links and connections, it is also a tool for customizing response by its support for modifying connection properties. In this way it can serve both the high-level systems perspective and the low-level functional view of mapping. Though designed for musical devices, the API's loose framework could readily be applied to any type of multimedia system. libmapper is an extremely powerful, flexible tool and requires a user interface that can elegantly deploy its full range of capabilities.

% section libmapper_bindings (end)

%%%%%%%%%%%%%%%%%%%%%%%%%%%%%%%%%%%%%%%%%%%%%%%%%%%%%%%%%%%%%%%%%%%%%%%%%%%%%%%%%%%%%%%%%%%%%%%%%%%%%%%%%%%%%%%%%%%%%%%%%%%%%%%%%%%%%%%%%%%%%%%%%%%%%%%%%%%%%%%%%%%%%%%%%%%%%%%%%%%%%%%%%%%%%%%%%%%%%%%%%%%%%%%%%%%%%%%%%%%%%%%%%%%%%%%%%%%%%%%%%%%%%%%%%%%%%%%%%%
\section{Prior interfaces for libmapper} % (fold)
\label{sec:prior_interfaces_for_libmapper}

	\subsection{Maxmapper} % (fold)
	\label{sub:maxmapper}

	A list-style interface described in detail is section \ref{sub:ListView}.
	
		\subsubsection{Limitations of Max/MSP}
		%breaks, does not scale well
	%no resizing (1260 by 777 px)
	%all tabs
	%lists at the top
	% subsection maxmapper (end)

	\subsection{Vizmapper} % (fold)
	\label{sub:vizmapper}
	
	% no resizing
	% connection editing cumbersome
	% subsection vizmapper (end)

	\subsection{Webmapper} % (fold)
	\label{sub:webmapper}
	
\begin{figure}[ht]
	\centering
	%\scalebox{0.42}{
	\includegraphics[width=1\textwidth]%
		{figures/webmapper}
	\caption{The webmapper interface}
	\label{fig:webmapper}
\end{figure}

Work on this project began with a moderately featured, little used GUI for libmapper known as ``webmapper.'' Webmapper was created at IDMIL as an attempt to replace the Max/MSP GUI as the result of limitations described in section \ref{sub:maxmapper}, the principle among which being the cross-platform incompatibility of Max/MSP. It was thought that a browser-based approach would greatly simplify the process of creating cross-compatibility with all major operating systems and even mobile devices. 

Webmapper utilizes the Python bindings for libmapper by registering an administrative monitor to communicate with a libmapper network. The monitor can create and modify connections or links, as well as query the network as to what devices, signals, links and connections are present. The webmapper code creates a simple HTTP server and attempts to open Google Chrome\footnote{Chrome Browser. [Online]. Available: \url{https://www.google.com/intl/en/chrome/browser/}. Accessed July 17, 2013} on the host computer. If Google Chrome is not present, the user must navigate directly to the server using the web address \url{localhost:50000}. The monitor communicates with the libmapper network and the local server, the browser is able to see messages the monitor ``posts'' to the server (such as `new device') and respond to them appropriately. The browser in turn can send messages to the server (like `connect') that will propagate up to libmapper itself, eventually resulting in a message cascading back down to the browser reflecting the change to the network (such as `new connection'). 

The interface itself is written for a web browser using the scripting language JavaScript\footnote{JavaScript | MDN. [Online]. Available: \url{https://developer.mozilla.org/en-US/docs/Web/JavaScript}. Accessed July 17, 2013} to control web-standard HyperText Markup Language (HTML) elements and Cascading Style Sheets (CSS). The JavaScript code stores four main data structures: devices, links, connections and signals. The code never directly modifies any of this data, and instead waits for messages relayed from libmapper. For example: if a user creates a new link, webmapper does not add the link directly to the links array but simply sends a message to the network. If it receives back a `new link' message, only then does it add the new link to the array. This is done to ensure that the data structures within webmapper always reflect what is actually present.

Figure \ref{fig:webmapper} displays the look of the interface before this project began. Users are able to perform all libmapper functions: connecting, linking and modifying connections, but only the simplest of feature sets is included. In order to form a connection the user must click on a source signal, click on a destination signal and then click on a button labeled ``connect.'' Many useful features of the Max/MSP interface, such as column headers, table sorting, drawing connections and search filtering are not present.

	% subsection webmapper (end)

% section prior_interfaces_for_libmapper (end)

%%%%%%%%%%%%%%%%%%%%%%%%%%%%%%%%%%%%%%%%%%%%%%%%%%%%%%%%%%%%%%%%%%%%%%%%%%%%%%%%%%%%%%%%%%%%%%%%%%%%%%%%%%%%%%%%%%%%%%%%%%%%%%%%%%%%%%%%%%%%%%%%%%%%%%%%%%%%%%%%%%%%%%%%%%%%%%%%%%%%%%%%%%%%%%%%%%%%%%%%%%%%%%%%%%%%%%%%%%%%%%%%%%%%%%%%%%%%%%%%%%%%%%%%%%%%%%%%%%
\section{Evaluation of libmapper variables} % (fold)
\label{sec:evaluation_of_libmapper_variables}

The list in table \ref{tab:metadata_types} is by no means a complete set, as libmapper may yet expand to include data like device position and users are able to tag devices/signals/links/connections with any extra data they may want.

A fourth data category \emph{boolean} has been added to specify data that only has two values (\emph{true or false}), as it is a common metadata feature. Boolean information is not covered in the Mackinlay paper. Going forward it will be treated more or less like ordinal data, as \emph{true} obviously has a relationship to \emph{false}, even though there is no quantitative value associated with them.

\begin{longtable}{l l l l}
\caption[libmapper metadata types]{libmapper metadata types} \label{tab:metadata_types} \\

	\hline\hline
	\textbf{Devices} & & \\
	\emph{quantitative} & \emph{ordinal} & \emph{nominal}\\
	\hline
	number of inputs & device ordinal & device name\\
		number of outputs \\
	ip address \\
	port \\ [0.7cm]

	\hline\hline
	\textbf{Signals} & & \\
	\emph{quantitative} & \emph{ordinal} & \emph{nominal}\\
	\hline
	length & direction & parent device name\\
	minimum value & & signal name \\
	maximum value & & data type (float, integer, etc.)\\
	sampling rate & & units \\ [0.7cm]

	\hline\hline
	\textbf{Links} & & \\
	\emph{quantitative} & \emph{ordinal} & \emph{nominal}\\
	\hline
	& & link name \\
	& & source device name \\
	& & destination device name \\
	& & scope \\ [0.7cm]

	\hline\hline
	\textbf{Connections} & & \\
	\emph{quantitative} & \emph{ordinal} & \emph{nominal} & \emph{boolean}\\
	\hline
	source minimum & instance number & boundary modes & mute\\
	source maximum & & connection mode & send as instance\\
	destination minimum & & destination data type \\
	destination maximum & & mute \\
	& & expression \\
	& & \\
\end{longtable}

% section evaluation_of_libmapper_variables (end)


%
\typeout{}
%!TEX root = ../thesis.tex
\chapter{Design \& Implementation}

	Development of a graphical user interface for libmapper creates a unique challenge. Obviously such an interface is a practical tool, and should function as such, yet it also must work in concert with DMIs which are inherently designed for creative use. For the purposes of this project, the assumed solution to this innate paradox is to provide the user with multiple independent modes of control.  libmapper itself is an extremely flexible API that makes few assumptions as to the network of devices and signals or how they are mapped. It is thus fitting that a GUI for libmapper would be equally as flexible. In lieu of a single perfect solution for network visualization and interactivity, providing users with various independent solutions provided a good compromise.

	Work began with the webmapper interface described in section \ref{sub:webmapper}. An MVC structure was built around the code in order to make the program more extensible and allow for the easy integration of multiple views. Missing features from Maxmapper are incorporated into the main view mode, known as the ``list'' view. That interface has been extended in various ways, taking advantage of the new code base. Two new view modes, the ``grid'' and ``hive'', both designed by Jonathan Wilansky at IDMIL, are integrated into the main GUI. Finally, the code has been compiled together as a standalone application, ready for wide distribution.

%%%%%%%%%%%%%%%%%%%%%%%%%%%%%%%%%%%%%%%%%%%%%%%%%%%%%%%%%%%%%%%%%%%%%%%%%%%%%%%%%%%%%%%%%%%%%%%%%%%%%%%%%%%%%%%%%%%%%%%%%%%%%%%%%%%%%%%%%%%%%%%%%%%%%%%%%%%%%%%%%%%%%%%%%%%%%%%%%%%%%%%%%%%%%%%%%%%%%%%%%%%%%%%%%%%%%%%%%%%%%%%%%%%%%%%%%%%%%%%%%%%%%%%%%%%%%%%%%%
\section{Development of a Flexible System} % (fold)
\label{sec:development_of_a_flexible_system}

Prior GUIs for libmapper have been successfully used for some time, but all have failed to become a standard for the same reason: they cannot accommodate all possible use-cases of libmapper. List based views like the Max/MSP GUI and webmapper cannot show hierarchies while the cluster view implemented in vizmapper can be overly cumbersome for interaction with simple networks. Especially with so much work already completed on prior GUIs, it is more suitable to integrate different approaches into a single GUI than to begin work on some new, hopefully superior approach that would likely prove to be flawed like all that came before it. 

Interface integration is accomplished through an extremely simple approach: a drop-down menu is added to the upper left corner of the interface. Options on this menu represent different visualization modes available to the user. By selecting a new visualization mode the GUI drastically changes it appearance, replacing nearly every visual element in the display.
	%Needs to be adaptable, show any metadata

	\subsection{MVC architecture} % (fold)
	\label{sec:mvc_architecture}

Because a modular design is desired, the Model-View-Controller (MVC) metaphor for structuring software applications \cite{MVC_krasnerpope} is used as a general framework for structuring the application. In fact, the whole scale swapping in and out of independent visual modes would be straightforward implementation of MVC. Unfortunately, the \url{libmapper} $\rightarrow$ \url{python monitor} $\rightarrow$ \url{browser} implementation is slightly more complicated than as imagined by \citeANP{MVC_krasnerpope}. Figure \ref{fig:mapper_network} can be contrasted with (TODO). A few layers of abstraction are added to take into account the monitor, the network itself and control features independent to the view (see section \ref{sec:top_toolbar}), but the general MVC architecture is maintained.

\begin{figure}[!ht]
\centering
	\includegraphics[width=0.8\textwidth]{figures/mapper_network}
\caption{MVC hierarchy for the GUI and libmapper, blue arrows show propogation of network changes, dashed arrows denote messages requesting a network change.}
\label{fig:mapper_network}
\end{figure}


		\subsubsection{Independent communication}

First and foremost, it is essential that data on the screen should reflect data on the network. This is not entirely straightforward, as asynchronous messages are constantly being sent from the GUI to libmapper and vice versa. In a truly distributed system, data on the libmapper network will be constantly changing as other users add devices and modify mappings. As can be seen in figure \ref{fig:mapper_network}, the actual libmapper network, the displayed data and user interaction are very insulated from one another. For example, a user command to link two devices (in this case \url{source.1} and \url{dest.1}), the following message will be sent to the python monitors:

\url{ {"cmd":"link","args":["/source.1","/dest.1"]} }

Meaning: a linking command is sent to \url{source.1} and \url{dest.1} (the command itself is a python dictionary). After this, the display will not do anything, as the display has not yet been notified of the link. The monitor then relays this message to the network, using libmapper specific syntax. If the link is successful, the monitor receives notice, and sends a message to the main javascript file:

\url{("new_link", {"src_name": "/source.1", "dest_name": "/dest.1"}) }

This states that a new link has been formed between the source device \url{source.1} and the destination device \url{dest.1}. Only then does the GUI respond to the change on the network. Signal data itself is not available to the GUI in any way, as libmapper networks are designed to prevent this kind of bandwidth clutter \shortcite{new_libmapper}.

		\subsubsection{The model}

The model consists of an abstract copy of the network, residing on the local machine. Independent views can consult these data, but cannot directly modify it. Messages from the python monitor announce new links, modifications to connections, or any other changes on the network. All of these changes are recorded and reflected in the model. The model itself consists of four data structures.

\begin{itemize}
 	\item \textbf{Devices}: Storage of all present devices and device metadata.
 	\item \textbf{Links}: A record of all links presently on the network.
 	\item \textbf{Signals}: Keeps track of signals on the network, but only whichever signals are currently visible in the GUI. This is done to save bandwidth and processing power. View-controller pairs keep track of which devices are currently being viewed, and can ask for their child signals. 
 	\item \textbf{Connections}: All connections and connection metadata between signals currently in the model.
 \end{itemize} 

It is possible that previously viewed signals will persist in the model, but they and their connections will not be updated upon change.

		\subsubsection{View-controller pairs}

All interaction handlers\footnote{Response to mouse clicks and certain keypresses.} and visualizations are stored in modular, view-controller pairs, as recommended by \citeN{MVC_krasnerpope}. Each view controller pair corresponds with a single view mode. Pairs can have any combination of UI handlers and visual features, but must have the following four functions that are called by the main file on which the model is stored:

\begin{itemize}
	\item \url{view.initialize()}: Calls upon the view to create its visual elements and add 
	\item \url{view.get_focused_devices()}: Returns whichever devices are currently visible in the view. This is used for populating the signal and connection data structures.
	\item \url{view.cleanup()}: Causes the controller to remove all interaction handlers.
	\item \url{view.update_display()}: Called whenever the model changes. The view is not aware
\end{itemize}


		%initialize, cleanup, update_display
	
	% subsection mvc_architecture (end)

	\subsection{Top toolbar} % (fold)
	\label{sec:top_toolbar}

Certain tasks and information providing structures are sensible to include across visualization modes. In light of this, a static toolbar is presented at the top of the GUI. This toolbar contains all administrative controls and connection modification fields. 

\begin{figure}[!h]
\centering
	\includegraphics[width=1\textwidth]{figures/top_toolbar}
\caption{The upper toolbar}
\label{fig:toolbar}
\end{figure}

\begin{itemize}
	\item \textbf{Administrative controls}
	\begin{itemize}
		\item\emph{Load/Save buttons}: These elements respond to clicks and save and load mappings, as discussed in section \ref{sec:saving_and_loading}.
		\item\emph{Visual mode selection}: A drop-down menu containing all possible view modes (at the writing of this thesis: List, Grid and Hive).
		\item\emph{Refresh Button}: When clicked, all data residing on the GUI is erased and re-gathered. This is useful if the monitor somehow desynchronizes with the network.
	\end{itemize}

	\item \textbf{Connection modification}
	\begin{itemize}
		\item\emph{Connection mode selectors}: If a single connection is selected within the GUI, this array of buttons allows the user to choose between the available connection modes.
		\item\emph{Expression editor}: Here the user inputs a custom expression, if in ``Expr'' mode, in other modes this field displays the connection's expression but is not editable.
		\item\emph{Source range editor}: These two numbers reflect the maximum and minimum values of the input signal, is only editable in the ``Line'' connection mode.
		\item\emph{Destination range editor}: Same as above but for the destination signal. Due to boundary conditions these fields are useful in all modes.
		\item\emph{Boundary mode selectors}: Two buttons that cycle through five boundary modes for both the maximum and minimum destination value. A graphic exists to represent each mode.
	\end{itemize}
\end{itemize}

All interface features not present in the top toolbar are part of the current visualization mode and are placed into a ``container'' element below, occupying the remainder of the window.

	% subsection top_toolbar (end)



% section development_of_a_flexible_system (end)

%%%%%%%%%%%%%%%%%%%%%%%%%%%%%%%%%%%%%%%%%%%%%%%%%%%%%%%%%%%%%%%%%%%%%%%%%%%%%%%%%%%%%%%%%%%%%%%%%%%%%%%%%%%%%%%%%%%%%%%%%%%%%%%%%%%%%%%%%%%%%%%%%%%%%%%%%%%%%%%%%%%%%%%%%%%%%%%%%%%%%%%%%%%%%%%%%%%%%%%%%%%%%%%%%%%%%%%%%%%%%%%%%%%%%%%%%%%%%%%%%%%%%%%%%%%%%%%%%%
\section{Integration of Interface Features} % (fold)
\label{sec:integration_of_interface_features}

Development began by unifying features of the Maxmapper onto the Webmapper code. Webmapper was selected as a starting point because of cross-platform nature of a web-based implementation. The general two-table structure of Maxmapper and Webmapper created the first view of the interface, known as the ``list'' view.

	\subsection{Structure of the list view} % (fold)
	\label{sub:the_list_view}

The list view provides the most straightforward way to visualize and interact with libmapper. Two tables dominate the visible area, listing source elements on the left and destination elements on the right. B\'ezier curves form lines between associated list elements on each each list. Because these curves do not always represent the same data structures, the lines themselves are referred to as \emph{arrows} by the GUI code, and by this document.

\begin{figure}[ht]
\centering
	\scalebox{0.4}{\includegraphics{figures/list_view_all_devices}}
\caption{The list view with all devices selected}
\label{fig:list_view_all_devices}
\end{figure}

The view itself is divided into two major modes: ``All Devices'' and individual linked source devices. Switching between these modes is accomplished through tabs that appear at the top of the container, much like the tabs that appear in modern web browsers. In the All Devices tab, every device displayed on the network is listed in one of the two columns, as in \ref{fig:list_view_all_devices}. Source devices are listed in the left table, while the right table lists destination devices. Intermediate devices, such as implicit mappers described in \cite{interpolated_mappings}, will be listed in both tables. Here arrows represent links between devices. Currently the GUI provides users with names, the number of child signals, IP addresses and a port for every device. Since no connections or signals are displayed, most of the top bar (see section \ref{sec:top_toolbar}) is disabled in the All Devices tab. Saving and loading are also disabled.

\begin{figure}[ht]
\centering
	\scalebox{0.4}{\includegraphics{figures/list_view_single_link}}
\caption{The list view with device \textbf{testsend.1} selected}
\label{fig:list_view_single_link}
\end{figure}

The GUI draws a tab for every source device with at least one link to a destination device. Clicking on any of these devices will redraw both tables. The left table now shows all child signals for the selected source device while the right table displays child signals for every destination device linked to the selected device. 

	% subsection the_list_view (end)
	
	\subsection{Display libmapper metadata} % (fold)
	\label{sub:display_libmapper_metadata}

The original webmapper interface tables listing devices and signals have no headers. Without these queues, only a small amount of metadata is provided (see figure \ref{fig:webmapper}):

\begin{table}
	\centering
	\Tcaption{Metadata available in webmapper vs list view}
	\label{tab:webmapper_list_view_metadata}
		\begin{tabular}{l  l  |  l l }
		\hline\hline
		\textbf{webmapper}&&\textbf{list view}\\
		Devices&Signals&Devices&Signals\\
		\hline
		name&name&name&name\\
		IP address&data type&IP address&data type\\
		port&vector length&port&vector length\\
		&&number of inputs&units\\
		&&number of outputs&maximum value\\
		&&&minimum value\\
		\end{tabular}
\end{table}

To include a necessary feature of max mapper, column headers are added to the list view. Also included are the new pieces of device and signal metadata listed in table \ref{tab:webmapper_list_view_metadata}. Tables draw themselves with invisible extra columns, such that adding extra data can be easily accomplished. If a user embeds extra metadata not listed above onto devices or signals, that data will automatically be included in the table display.  

In general, the GUI tries to keep possible extensions to libmapper like this in mind. Very little is assumed about the network itself. In turn, the only device metadata that \emph{must} exist is the name and number of inputs/outputs, which is used to either display the device as a source or destination. For signals, the GUI takes vector length into account when deciding whether two signals are compatible and can be connected. However, disincluding length in the signal metadata will not result in an error.
	
	% subsection display_libmapper_metadata (end)

	\subsection{Visual feedback} % (fold)
	\label{sub:visual_feedback}
		% # of signals/etc.
		% muted lines
		% row striping

	% subsection visual_feedback (end)

	\subsection{Namespace filtering} % (fold)
	\label{sub:namespace_filtering}
		%hide unconnected signals
	% subsection namespace_filtering (end)

	\subsection{Draggable connections and links} % (fold)
	\label{sub:draggable_connections_and_links}
	
	% subsection draggable_connections_and_links (end)

% section integration_of_interface_features (end)

%%%%%%%%%%%%%%%%%%%%%%%%%%%%%%%%%%%%%%%%%%%%%%%%%%%%%%%%%%%%%%%%%%%%%%%%%%%%%%%%%%%%%%%%%%%%%%%%%%%%%%%%%%%%%%%%%%%%%%%%%%%%%%%%%%%%%%%%%%%%%%%%%%%%%%%%%%%%%%%%%%%%%%%%%%%%%%%%%%%%%%%%%%%%%%%%%%%%%%%%%%%%%%%%%%%%%%%%%%%%%%%%%%%%%%%%%%%%%%%%%%%%%%%%%%%%%%%%%%
\section{Extension of Control and Visual Elements} % (fold)
\label{sec:extension_of_control_and_visual_elements}

	\subsection{Keyboard shortcuts} % (fold)
	\label{sec:keyboard_shortcuts}

		%select all, large selections
	
	% subsection keyboard_shortcuts (end)

	\subsection{Window resizing} % (fold)
	\label{sec:window_resizing}
	
	% subsection window_resizing (end)

	\subsection{Variable line heights} % (fold)
	\label{sec:variable_line_heights}
	
	% subsection variable_line_heights (end)

	\subsection{Visual Redesign} % (fold)
	\label{sec:visual_redesign}
	
	% subsection visual_redesign (end)

	\subsection{Grid \& Hive views} % (fold)
	\label{sec:grid_&_hive_views}

\begin{figure}[ht]
\centering
	\scalebox{0.4}{\includegraphics{figures/grid}}
\caption{The grid view}
\label{fig:grid}
\end{figure}

\begin{figure}[ht]
\centering
	\scalebox{0.4}{\includegraphics{figures/hive}}
\caption{The hive view}
\label{fig:hive}
\end{figure}
	
	% subsection grid_&_hive_views (end)

% section extension_of_control_and_visual_elements (end)

%%%%%%%%%%%%%%%%%%%%%%%%%%%%%%%%%%%%%%%%%%%%%%%%%%%%%%%%%%%%%%%%%%%%%%%%%%%%%%%%%%%%%%%%%%%%%%%%%%%%%%%%%%%%%%%%%%%%%%%%%%%%%%%%%%%%%%%%%%%%%%%%%%%%%%%%%%%%%%%%%%%%%%%%%%%%%%%%%%%%%%%%%%%%%%%%%%%%%%%%%%%%%%%%%%%%%%%%%%%%%%%%%%%%%%%%%%%%%%%%%%%%%%%%%%%%%%%%%%
\section{Other GUI features} % (fold)
\label{sec:other_gui_features}

	\subsection{Saving \& Loading} % (fold)
	\label{sec:saving_and_loading}
	
	% subsection saving_&_loading (end)

	\subsection{Creation of a standalone \& distribution} % (fold)
	\label{sec:creation_of_a_standalone_and_distribution}
	
	% subsection creation_of_a_standalone_&_distribution (end)

% section other_gui_features (end)





%
\typeout{}
%!TEX root = ../thesis.tex
\chapter{Applications \& Discussion}

This chapter presents a discussion of MapperGUI's software design and its consequences for musical mapping, as well as revisions made to the code since the initial release. The interface's features are explored in an attempt to evaluate the successes and failures of the design. Feedback from users was gathered throughout the project and through informal interviews after the software's release. This feedback is summarized and presented here. A modification to the code, motivated by feedback from users, is also described. MapperGUI is then compared to similar interfaces, analyzing especially for new features that could be incorporated into our flexible framework. Finally, the system is evaluated overall with respect to the project's initial goals.


%%%%%%%%%%%%%%%%%%%%%%%%%%%%%%%%%%%%%%%%%%%%%%%%%%%%%%%%%%%%%%%%%%%%%%%%%%%%%%%%%%%%%%%%%%%%%%%%%%%%%%%%%%%%%%%%%%%%%%%%%%%%%%%%%%%%%%%%%%%%%%%%%%%%%%%%%%%%%%%%%%%%%%%%%%%%%%%%%%%%%%%%%%%%%%%%%%%%%%%%%%%%%%%%%%%%%%%%%%%%%%%%%%%%%%%%%%%%%%%%%%%%%%%%%%%%%%%%%%
\section{User Feedback} % (fold)
\label{sec:user_feedback}

The entire MapperGUI project began with user feedback from prior GUIs for libmapper. Throughout the design process functional versions of MapperGUI were provided to libmapper users. Their feedback was crucially important to the evolution of the software. After the first official release of MapperGUI, long-term users were informally interviewed. These users were questioned specifically as to the projects with which utilized MapperGUI in an attempt to learn more concretely about the variety of use-cases for the software.

Even at this early stage of release, users have already incorporated MapperGUI into a wide variety of projects. This reflects our initial assumptions (based on experiences with prior GUIs) that a successful GUI must be flexible. Throughout development MapperGUI was used as an experimental tool and aid in designing DMIs. MapperGUI was used in concert with motion capture systems, vibrotactile feedback and even loaded onto a Raspberry PI\footnote{Raspberry Pi | An ARM GNU/Linux box for \$25. Take a byte! [Online] Available: \url{http://www.raspberrypi.org}. Accessed August 1, 2013}. During this whole process users encountered problems, had ideas for extensions and used the GUI in ways we could have never imagined.



%for response to vibrotactile feedback in a motion-capture study. It was also used in a motion capture setting for the design of an interactive audio installation. In the latter situation MapperGUI was required to handle many signals per single device, as each person in the room required 25 three-dimensional markers (75 signals total). A DMI designer has been using MapperGUI to test mappings for her input device with a single sound synthesizer, each containing about 30 signals.

%MapperGUI is being used as a development tool as well. A programmer is attempting to build a software bridge between libmapper and the Arduino\footnote{TODO}. She uses MapperGUI to test the robustness and effectiveness of the software, and has even successfully loaded the GUI onto a raspberryPI\footnote{TODO}

%\begin{table}
%\begin{center}
%\begin{tabular}{l p{5cm} p{5cm}}
	%\hline\hline
	%user&use case&concerns\\
	%\hline
	%Mailis&Intonespacio&saving and loading\\
	%Hakon&Experimenting with vibrotactile feedback and motion capture systems&Switching between various mappings\\
	%Clayton&An interactive space using motion capture&Reliability of network\\
	%Julie&libmapper code for firmata&Speed of function (she's using a rasperry Pi)\\
	%\hline
	%Andrew Stuart&teaches class with libmapper&\\
	%Gestes (Marlon)&Performance, etc.&Hide unconnected\\
%\end{tabular}
%\end{center}	
%\end{table}
%
%Hakon Knutzen, Mailis Rodrigues, Clayton Mamedes, Julie Ren\'e
		%%Mailis, Clayton, Hakon, Julie


	\subsection{General feedback} % (fold)
	\label{sub:general_feedback}

Most of our users had used libmapper previously, and had attempted to compile and use the library from scratch. Many commented on how well MapperGUI lowered the barriers to entry for non-technical users. Users who had never used libmapper before pointed out how much time had been saved in their work flow, versus when they used to hard-code mappings.

The best reviewed feature of MapperGUI was the automatic linear scaling control found in the top bar. Some users previously needed to detect signal minima and maxima by hand, then directly calculate and apply a linear scaling functions. With MapperGUI the task is trivially easy: one must simply enter the desired destination range and set the connection to the \emph{Calibrate} mode. Most of the ``magic'' in this feature is the result of the libmapper API, but providing users access through an easy-to-use GUI is also important. One user expressed frustration because she was not aware this feature existed, and instead continued to painstakingly condition her signals in Max/MSP. She was very impressed over how much time was saved by switching this workload to libmapper and MapperGUI.

Use of the other connection modes was rare. Users found the expression input box hard to use. Directly calculating the appropriate mathematical expression was seen as too abstract. This is a sensible problem to have, as difficult text-based input is precisely the thing that MapperGUI is designed to avoid. One user suggested a two-dimensional graphical tool, showing the transposition from input to output, to help with this task. 

Some users requested for signal values themselves to be available in MapperGUI. This would create a lot of bandwidth clutter, as all devices would need to constantly send signal data to the GUI. It was suggested the user should be able to query signal data by clicking or placing the mouse cursor over signal names.
	
	% subsection other_feedback (end)

	\subsection{Saving \& loading} % (fold)
	\label{sub:saving_and_loading}

Nearly all users made use of the saving and loading features in some way. For both experimental and design-based setups, returning to prior mappings is very useful, as it avoids the tedium of performing the same tasks repeatedly.

We received criticism for the na\"ive loading system. One user found it tedious that mappings would accumulate when loading multiple. He required rapid switching between the same few mappings for his experiment. Once these mappings were created there was little that needed to be done to modify them. For his experiment it became tedious to erase a previous mapping before loading a new one. Obviously in a live-performance context the amount of delay that would be created with this kind of task would be unacceptable. 

Another user wished to switch between mappings in his work, but required some kind of intermediate space between the states. Ideally loading would have the option of blending between two mappings, such that the transition is not too harsh for the audience. To maintain this functionality, the actual saving and loading of patches was transferred to Max/MSP for this project.

In a situation with many devices of the same class, loading a single mapping can be somewhat absurd. Because each connection will be loaded $m*n$ times, where $m$ is the number of similarly named input devices, and $n$ is the number of relevant output devices, certain simple mappings can result in hundreds of unwanted connections when a mapping is loaded. Perhaps some kind of staging area wherein the user must explicitly for which devices he or she wishes to load a mapping could solve this problem.

Another user asked for some kind of mapping preset that could be created and loaded whenever the program is opened. This way if the same experiment or performance is conducted repeatedly the user would simply need to launch MapperGUI and get started.
	
	% subsection saving_&_loading (end)

	\subsection{Reliability \& responsiveness} % (fold)
	\label{sub:reliability_and_responsiveness}

Multiple users commented on the frustrating nature of interacting with MapperGUI when it became out of sync with the libmapper network. As one user stated, ``The program is not useful if you do not \emph{trust} the display.'' In this way small errors, devices not appearing, signals not accepting connections, delays in operations, etc. become a very big problem for user satisfaction with MapperGUI. Users reviewed the refresh button very favorably. If something seemed amiss with the GUI or the network, and refreshing the display solved the problem, then trust in the display was restored.

Some problems were due to errors in the libmapper code and were out of the control of MapperGUI. Others were created when MapperGUI code started to make assumptions about the libmapper network. For example, with the drag-to-connect gesture, originally the drawn arrow persisted upon release of the mouse button. MapperGUI assumed that a connection would be made and kept the line to avoid delays. Of course, occasionally the signals were \emph{not} connected, due to dropped messages or incompatibility. In this circumstance the faulty arrow, representing nothing, became very confusing. Due to negative feedback the code was changed such that the drawn arrow disappears immediately after the drawing gesture. If the connection is successful, it is redrawn. This results in a slight flicker (as the arrow is erased and re-drawn), but this was much more popular than the potential erroneous arrows persisting in the display.

Some heavy operations, like scrolling and forming multiple connections, could create significant delays (on the order of a few seconds) in MapperGUI. Users responded very negatively to such delays, as they were accustomed to computer programs responding much more quickly. Generally multi-second delays were thought to be errors, thus reducing the user's trust in the application. In Section \ref{sec:testing_program_responsiveness} we explore solutions to this problem.
	
	% subsection reliability_and_responsiveness (end)

	\subsection{Effectiveness of alternate views} % (fold)
	\label{sub:effectiveness_of_alternate_views}

GridView and HiveView have only recently been recently included into the program. As a result most of our users are much more familiar with ListView. Users reported that while the alternate views were interesting, ListView was the most straightforward for creating mappings. It was reported that GridView was useful once most of the mapping was completed, as one could notice patterns that were not apparent in ListView, and alter mappings to reflect the findings. The limited functionality of HiveView meant that to most users it was simply a visualization tool. Also it was extremely common among our test users for use-cases to include very few devices with many connections, so the ``whole-network'' view in HiveView was not advantageous.
	
	% subsection effectiveness_of_alternate_views (end)
	
% section user_feedback (end)

%%%%%%%%%%%%%%%%%%%%%%%%%%%%%%%%%%%%%%%%%%%%%%%%%%%%%%%%%%%%%%%%%%%%%%%%%%%%%%%%%%%%%%%%%%%%%%%%%%%%%%%%%%%%%%%%%%%%%%%%%%%%%%%%%%%%%%%%%%%%%%%%%%%%%%%%%%%%%%%%%%%%%%%%%%%%%%%%%%%%%%%%%%%%%%%%%%%%%%%%%%%%%%%%%%%%%%%%%%%%%%%%%%%%%%%%%%%%%%%%%%%%%%%%%%%%%%%%%%
\section{Testing program responsiveness} % (fold)
\label{sec:testing_program_responsiveness}

Extension of interface features discussed in Section \ref{sec:extension_of_control_and_visual_elements} leads to some control possibilities that could be difficult for the GUI to handle. Addition of shortcut keys for connecting, linking, disconnecting and unlinking, as well as the ability to select multiple devices and signals at once allows users to create and delete hundreds of connections with a single key press. Na\"{i}ve saving and loading produces to situations where dense mappings will accidentally be applied to several instruments at once.

Though the ``update display'' model works extremely well for code modularity, it generates awkward situations when dealing with massive network operations. Since the system updates the entire display with each change to the network, deleting 100 links (if the user is clearing a large network) results in 100 independent \url{delete_link} messages arriving at the monitor. For each one of these messages, the display will fully update. In the case of ListView, all arrows will be cleared, and redrawn with one fewer present (as if the links are being deleted one-by-one). In total 4950 arrow drawing operations\footnote{$99 + 98 + 97 + ... + 2 + 1 = \frac{99*100}{2}$. Note that $\frac{n*(n+1)}{2}$ arrows will be drawn for any $n$ number of connections or links.} will occur, resulting in significant delay. 

As reported by users in Section \ref{sub:reliability_and_responsiveness}, any GUI operation that takes more than a few seconds, without some kind of visual feedback (like a ``loading'' bar), leads to frustration and mistrust of the program. Obviously, if the GUI is going to support these kinds of massive network manipulations, there needs to exist some way to keep them under control.

	\subsection{Rate limiting functions} % (fold)
	\label{sub:rate_limiting_certain_functions}

In order to prevent thousands of unnecessary, display re-draws, a ``waiting'' period was added to certain critical functions. This operating systems process is described in \shortciteN{os_concepts}. Essentially certain functions no longer execute in the code immediately once called, instead, a delay timer starts. If the function is called again during this delay, the delay timer simply restarts. The function is only executed once the delay timer finishes. This way, if a function is called 100 times simultaneously, it will only execute once after a short delay. Figure \ref{fig:waiting_period} shows the effect of the waiting period if the function is called a single time, and if it is called once during the delay.

\begin{figure}
	\centering
		\includegraphics[width=1\textwidth]{figures/waiting_period}
		\caption{Illustration of a delayed function.}
		\label{fig:waiting_period}
\end{figure}

Two functions are limited in the GUI: \url{view.update_display} for all views and \url{update_arrows} for ListView. \url{update_display} is common to all views, and the massive network changes described above can result in many processor intesive functions to run needlessly. With the \url{update_arrows} function, operations that do not result in changes to the network (scrolling, changing tabs) require constant re-drawing.

Exactly how much time this delay should be set to is not obvious. If the delay is too short it is possible for massive network operations to still call the function many times if they arrive asynchronously. A too-long delay means that users may notice for simple actions, like scrolling with a single arrow (causing the scrolling to appear jerky). Another consequence of a long delay is that a process which calls the delayed function at a regular interval could continuously restart the delay. In this case the function will \emph{never} execute, a situation known as ``starvation'' \shortcite{os_concepts}.

After some informal tests of delays between 17 and 1000 milliseconds, a delay of 33 milliseconds was selected for both functions. Substantial improvement in execution speed was observed for even very short delays, as often hundreds of function calls would reach the \url{view.update_display} virtually simultaneously. With delays closer to one second there was little improvement in response to massive network operations, and the delay itself became noticeable (especially with scrolling operations). 33 milliseconds is in the range where nearly every operation results in just a single function being executed, but is also an imperceptible delay to a human user. The number 33 itself was selected because it is the length of two screen refreshes on a 60 Hz display (a measure recommended in \shortciteNP{os_concepts}).
	
	% subsection rate_limiting_certain_functions (end)

% section testing_program_responsiveness (end)

%%%%%%%%%%%%%%%%%%%%%%%%%%%%%%%%%%%%%%%%%%%%%%%%%%%%%%%%%%%%%%%%%%%%%%%%%%%%%%%%%%%%%%%%%%%%%%%%%%%%%%%%%%%%%%%%%%%%%%%%%%%%%%%%%%%%%%%%%%%%%%%%%%%%%%%%%%%%%%%%%%%%%%%%%%%%%%%%%%%%%%%%%%%%%%%%%%%%%%%%%%%%%%%%%%%%%%%%%%%%%%%%%%%%%%%%%%%%%%%%%%%%%%%%%%%%%%%%%%
\section{Comparison to Similar Interfaces} % (fold)
\label{sec:comparison_to_similar_interfaces}

Other systems exist to help non-programmers map control inputs to sound synthesis parameters. This section compares this research to these related works.

\begin{figure}
	\centering
		\includegraphics[width=\textwidth]{figures/junXion_v4}
		\caption{STEIM's JunXion software}
		\label{fig:junxion}
\end{figure}

The Studio for Electro-Instrumental Music (STEIM) distributes JunXion \cite{junxion}, a software application for controlling MIDI and OSC-based systems. JunXion automatically detects input devices like computer mice and USB video-game controllers. The user is able to drag any child signals from these controllers onto one of 25 possible inputs.  From there the user can switch to the ``actions'' tab, where destinations and connection properties can be customized. Connection properties are stored in groups that populate drop-down menus in the central column. JunXion features a very interesting ``state'' system similar to MapperGUI's saving and loading. Once a successful mapping is created, users can change the state, starting a new mapping. With multiple mappings, users can quickly switch between states. JunXion also has a very interesting graphical signal conditioning editor. The program presents a two dimensional field and the user can draw, generate curves and set bounds. Incorporating such a feature into MapperGUI would assist users who were unimpressed by textual expression input.

\begin{figure}
	\centering
		\includegraphics[width=\textwidth]{figures/osculator}
		\caption{The OSCulator interface}
		\label{fig:osculator}
\end{figure}

The OSCulator system \cite{osculator} is very similar to JunXion. Compatible controllers appear automatically and can be mapped to MIDI or OSC signals. OSCulator also relies on a drop-down menu based interface for selecting where and how the output will be routed. As in JunXion the idea of a ``connection'' is not emphasized, instead a MIDI or OSC message is simply sent on a specific channel (the receiving end must be specifically notified on which channel to receive messages). As can be seen in Figure \ref{fig:osculator}, OSCulator displays a real-time oscilloscope like visualization for selected signals. A similar feature would help MapperGUI with visual feedback, which was a desire of some users.

The Eaganmatrix \cite{eaganmatrix} partly inspired GridView in MapperGUI. The signals of a single control and synthesis device are displayed on the x and y axes of a grid display. Connections between the two are made by clicking on the intersections. The Patchage interface \cite{patchage} contains uses an interaction very similar to ListView, where objects containing lists of signals can be connected by dragging gestures. Max/MSP and an Integra Live \cite{integra} also feature this interaction, but neither are necessarily for creating mappings. 

	% subsection other_similar_interfaces (end)

% section comparison_to_similar_interfaces (end)

%%%%%%%%%%%%%%%%%%%%%%%%%%%%%%%%%%%%%%%%%%%%%%%%%%%%%%%%%%%%%%%%%%%%%%%%%%%%%%%%%%%%%%%%%%%%%%%%%%%%%%%%%%%%%%%%%%%%%%%%%%%%%%%%%%%%%%%%%%%%%%%%%%%%%%%%%%%%%%%%%%%%%%%%%%%%%%%%%%%%%%%%%%%%%%%%%%%%%%%%%%%%%%%%%%%%%%%%%%%%%%%%%%%%%%%%%%%%%%%%%%%%%%%%%%%%%%%%%%
\section{Evaluation of Goals} % (fold)
\label{sec:evaluation_of_goals}

A set of goals for the software was established at the beginning of this document. These were to create an interface for libmapper that was easy to use for non-programmers. 
	
	%Why is what what on the top bar`'

	%maybe write about different views here?

% section evaluation_of_goals (end)





%
\typeout{}
\chapter{Conclusions \& Future Work}

\section{Summary and Conclusions}

\section{Future Work}
%%==========



%%========== Appendices
%\appendix
%
%%==========
%\typeout{}
%\input{A-A}

%========== Bibliography
\typeout{}
%\begin{singlespace}
    \renewcommand\refname{References}
    \nocite{*}
    \bibliography{../citations/references}%,../citations/other_references}
    \bibliographystyle{Style/ichago}
%\end{singlespace}

\end{document}
