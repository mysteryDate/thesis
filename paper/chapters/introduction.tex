%!TEX root = ../thesis.tex

\chapter{Introduction \& Motivation}
%%%%%%%%%%%%
%  (start with some kind of epigraph, maybe Tufte?)
%  cite: maybe Marcelo's paper on mapping, DOT, libmapper
%	look at Joe's 1-3
%	1: separate parts
%	S. Mann, “Natural interfaces for musical expression: Physiphones and a physics-based organology,” in Proceedings of the 2007 Conference on New Interfaces for Musical Expression (NIME-07), (New York, USA), pp. 118–123, 2007.
%	2: controller any arbitrary shape
%	E. R. Miranda and M. M. Wanderley, New Digital Instruments: Control and Interaction Beyond the Keyboard, vol. 21 of The Computer Music and Digital Audio Series. Middleton, Wisconsin, USA: A-R Editions, Inc., 2006.
%	3: mapping
%	J. Ryan, “Some remarks on musical instrument design at STEIM,” Contemporary Music Review, vol. 6, no. 1, pp. 3–17, 1991.
%	probably use marcelo's paper instead
%%%%%%%%%%%%

\begin{quote}
``In order that our tools, and their uses, develop effectively: (A) we shall have to give still more attention to doing the approximately right, rather than the exactly wrong...'' \cite{tuckey}
\end{quote}

Throughout the vast majority of human history the term ``musical instrument'' has signified both the physical object with which the musician interacted \emph{and} the direct source of the sound created: a violin with vibrating strings, a reeded saxophone, a timpani with its membrane, etc.  With the advent of electronic sound in the late 19\textsuperscript{th} century, it became possible for interactive objects to be separated from the sound producing devices they control \cite{chadabe}.
As technological development progressed, so did the capacity to divide musical instruments into independent parts. With digitization it is now not only possible to arbitrarily connect a control element to any sound synthesis dimension, but also to modify this association according to the whims of the user. Since mechanical linkages are no longer necessary in the design of musical instruments, control surfaces can, and often do, take on a variety of wild and arbitrary shapes and modes of interaction.\footnote{International Conference on New Interfaces for Musical Expression. [Online]. Available: \url{http://www.nime.org/}. Accessed June 23, 2013.}
All that is necessary for this process is for control devices to output some kind of electronic signal that other, sound-producing instruments can accept. With no obvious means of implementation, the success or failure of these new digital musical instruments (DMIs) often depends on how artfully their output signals are ``mapped" to synthesis parameters.

More and more frequently, the mapping itself becomes a part of the expressive element of a musical work \shortcite{mappingstrategies},
as it associates itself with both composition and performance with certain DMIs. Thus is becomes necessary for mapping to be dynamic and interactive: sometimes poured over in composition studios, or sometimes edited mid-piece. Musicians are not necessarily computer programmers, thus ideally the act of mapping should not require computer expertise. This means that on top of the low-level layer of interactive mapping (simply instructing a machine to connect signals to others in specific ways), there needs to exist an interface to make such an activity easy, logical, intuitive and in line with the artistic process.

As the actual act of mapping is as expansive and nebulous as the instruments it hopes to assist, thus the design of such a mapping interface presents many interesting challenges. Due to the tremendously wide variety of possible use cases, several seemingly contradictory goals emerge: What is the best way to visually represent complex musical networks while simultaneously allowing for the user to easily manipulate them? How can systems with many devices and signals be well represented while still allowing in-depth control of small networks? How can an interface be transparent to non-technical users while still accommodating all possible functionality that advanced users may wish to use? 

%Though it may not be possible to find a perfect solution to all of the above questions, it \emph{is} feasible address each in turn and accept the best available compromise. Overall it is simply necessary to wonder: What are useful features of a graphical interface for musical mapping?

\section{Context and Motivation}

The world of digital musical instruments is still dominated by keyboard type input devices. Though many novel DMIs currently exist (and many more are being created) these devices are usually unique and often difficult to use without their creator being present \cite{squeezevox}.
Since mapping is such an important feature of DMIs, a means of transparently editing them could inspire more people to use novel musical controllers. In response to this challenge, libmapper, a tool for collaborative mapping, was created at the Input Devices and Music Interaction Laboratory (IDMIL).

In its most basic state, libmapper takes the form of an application programming interface (API). APIs are primarily a means for different pieces of computer software to communicate with one another. The only possible way to communicate directly with the libmapper API is through coded text. For example, figure \ref{fig:libmapper_code} causes a synthesizer to announce itself and begin communicating with other devices on a libmapper-enabled network \shortcite{malloch}.

\begin{figure}[h!]
\begin{lstlisting}[]
#include <mapper.h>
mapper_admin_init();
my_admin = mapper_admin_new("tester", MAPPER_DEVICE_SYNTH, 8000); 
mapper_admin_input_add(my_admin, "/test/input","i")) 
mapper_admin_input_add(my_admin, "/test/another_input","f"))

// Loop until port and identifier ordinal are allocated. 
while ( !my_admin->port.locked	|| !my_admin->ordinal.locked )
{
	usleep(10000); // wait 10 ms 
	mapper_admin_poll(my_admin);
}

for (;;) 
{
	usleep(10000);
	mapper_admin_poll(my_admin); 
}
\end{lstlisting}
\caption{A sample of libmapper code}
\label{fig:libmapper_code}
\end{figure}

This is obviously inaccessible to users who do not have the time or desire to read through documentation files, or those who have no knowledge of programming semantics. A steep learning curve is especially a problem for a network tool like libmapper: because it is primarily a means of communication between instruments, it can only be successful if it is widely adopted. A libmapper-enabled controller will only be useful if many high quality libmapper synthesizers exist. In turn, synthesizer makers will only have incentive to incorporate libmapper into their designs if there are already controllers that use the system. 

An API can be contrasted with a graphical user interface (GUI), an interface that contains abstractions on top of the raw code. These abstractions can be features like buttons, menus, visual representations of data, etc. In general, GUIs are designed to be familiar to those who have used digital devices in the past, and thus easy to learn and use. Two GUIs have been created for libmapper (see section \ref{sec:priorGUIs}): a basic interface built in Max/MSP\footnote{MAX: You make the machine that makes your music. [Online]. Available: \url{http://cycling74.com/}. Accessed June 17, 2013} and vizmapper \cite{vizmapper}, a more abstract representation of a libmapper network. Both of these GUIs have their strengths, yet neither adequately meets the full range of possible use cases for libmapper. A more flexible approach is required if the GUI is to be usable in situations with hundreds of signals, transparent for systems with multi-leveled hierarchical devices, intuitive during performances where devices output light and haptic feedback as well as sound, and responsive for tasks where speed of manipulation is an absolutely necessity. 

With such an interface in place, libmapper can greatly expand its user base. As a result, more controller and synthesizer designers may choose to incorporate libmapper into their devices, and in turn these devices will be easier to learn and use. Hopefully the end result will be greater adoption of non keyboard-based DMIs in the electronic music community.

%Possible use cases: Performance, composition, demonstration (of a new DMI), experimentation
%	Actuator need not necessarily to be SOUND devices, but could be vibrotactile feedback, light projectors, 

\section{Project Overview}

The focus of this project is to create a graphical user interface for libmapper. This interface aims to be flexible and intuitive, simultaneously allowing for useful control of the full range of possible libmapper networks while also not intimidating non-technical users with complexity. The presupposed solution to this problem is to provide users with multiple independent modes of viewing and interacting with the network. Certain view modes can excel in providing precise control, while others can help users understand the structure of complex networks. The idea is to provide multiple imperfect solutions to an unsolvable problem, so that each can be ``...approximately right, rather than exactly wrong'' \shortciteA{tuckey}.

This project was structured in four major, non-sequential parts: a review of prior visualized mapping interfaces, the integration of presently available GUIs for libmapper, the extension of interface features and the collection of user feedback. Results of the research phase informed implementation and are presented here. Development began by updating a cross-platform implementation of the current Max/MSP-based GUI, while integrating functionality from vizmapper. New view modes were integrated into design while refining functionality of the previous ones. Throughout the design process, the GUI was provided to potential users who gave feedback on the strengths, weaknesses and potential avenues for improvement.

\section{Thesis Overview}

The remainder of this document is organized as follows. Chapter 2 outlines concepts necessary for providing context for this thesis project. A wide variety of domains inform the creation of a musical mapping interface. Special attention is paid to mapping theory, data visualization, relevant existing user interfaces, user centric design techniques and specifics of libmapper itself. Chapter 3 describes the design and implementation of the libmapperGUI. This chapter presents design decisions made, technical details of implementation and how the user-centric approach informed the process. Chapter 4 evaluates results, both on the empirical level of software performance as well as qualitative user feedback. Finally, Chapter 5 presents conclusions of the work and suggests further developments for the software.

\section{Contributions}

The contributions of this thesis are: the exploration of issues related to user interface design for musical mapping networks, the design and implementation of an interface for libmapper that aims to improve on usability and flexibility of the system, and this thesis document, which describes the research and development therein.

