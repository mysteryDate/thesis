%!TEX root = ../thesis.tex
\chapter{Background}

Dynamic mapping is becoming an increasingly important requirement for digital musical instruments. This chapter surveys currently available tools that allow for manipulation of musical and non-musical networks in real time. The first section presents a review of mapping itself, both from a theoretical and a musical standpoint. This portion also introduces the libmapper application programming interface. The second section reviews relevant work in the visual representation of information. The third section of this chapter describes applicable techniques in user interface design. Finally, a review of user interfaces for mapping is presented.

\section{Mapping}

	\subsection{Mapping Theory}

		\begin{itemize}
			\item many2one/one2one/one2many/many2many, certainly can cite a gestures class readgin %tocite{}
			\item Cite vijay about the inputs-outputs/outputs->inputs/sources->destinations thing
		\end{itemize}


	\subsection{Mapping for Digital Musical Instruments}
	\subsection{libmapper}
		joe's libmapper paper: \shortcite{malloch}
		joe's other paper? (earlier), his master's thesis
		\subsubsection{Open Sound Control}
		OSC: \shortcite{osc}
		\subsubsection{Structure of libmapper Networks}
		\subsubsection{Control of libmapper Devices and Signals}
	\begin{enumerate}
		\item GDIF: \shortcite{GDIF} Describes the namespace that libmapper uses, ``Ideally, it should be possible to store all sorts of data from various commercial and custom made controllers, motion capture and computer vision systems, as well as results from differ- ent types of gesture analysis, in a coherent and consistent way. This would make it possible to use the information with different software, platforms and devices, and also allow for sharing data between research institutions.'' 
		\item disembodied performance
		\item Wanderley's mapping paper \shortcite{wanderley} 
		\item Jamoma \shortcite{jamoma}
		\item surely some other stuff from class
	\end{enumerate}

\section{Data Visualization}
	\subsection{Graphical Perception}
		\subsubsection{Heirarchical Structures}
		\subsubsection{Dense Information}
	\subsection{Visualization Techniques}
		\subsubsection{Filtering}
		\subsubsection{Spark Lines}
		\subsubsection{Dash Plots}
	\subsection{Visualization Systems}
		Allosphere?, Braun Braun: view OSC data flows \shortcite{braun}, HEB?
	\begin{enumerate}
		\item Allosphere? :\shortcite{allosphere}
		\item Heirarchical edge bundling: \shortcite{HEB}
		\item Tukey: \shortcite{tuckey}
		\item Envisioning information: \shortcite{tufte1}
		\item Beautiful Evidence: \shortcite{tufte2}
		\item The other Tufte book I have at home.
		\item OSC data flows with Braun \shortcite{braun}
	\end{enumerate}

\section{User Interface Design}
	\subsection{A Brief History of Electronic User Interfaces}
	\subsection{Task Analysis}
	\subsection{Recall and Recognition?}
	\subsection{Collaborative Network Interfaces}
		MPG Care Package \shortcite{MPGcarepackage}
	\subsection{The Model-View-Controller Architecture}
		MVC Krasner Pope \shortcite{MVC_krasnerpope}
	\subsection{User Centric Design}
		Organizational context \shortcite{usd}
		Usability testing \shortcite{usd_corry}
		Information professionals \shortcite{usd_schulze}
	\begin{enumerate}
		\item Inclusive interconnections \shortcite{inclusiveinterconnections}
		\item Sense Stage \shortcite{senseStage}
	\end{enumerate}

\section{Relevant User Interfaces}
	\subsection{Junxion}
		Junxion \shortcite{junxion}
	\subsection{Osculator}
		Osculator: mapping OSC stuff \shortcite{osculator}
	\subsection{Other Similar Interfaces}
		Integra \shortcite{integra}
		Eaganmatrix: GRID VIEW! \shortcite{eaganmatrix}
		Patchage: a linking, dragging, connecting interface \shortcite{patchage}
	\subsection{Prior Interfaces for libmapper} \label{sec:priorGUIs}
		Vizmapper \shortcite{vizmapper}
	

\section{Summary}
	