%!TEX root = ../thesis.tex
\chapter{Conclusions \& Future Work}

This chapter summarizes the work presented this thesis, presents conclusions and summarizes possible avenues for further research.

\section{Summary and Conclusions}

This thesis began by exploring issues relevant to musical mapping interfaces. DMI designers typically hard-code mappings into their designs, making collaboration, cross-compatibility and modification difficult. Certain tools exist to aid these designers and their users, but they are often inaccessible laypersons. Our work was motivated by this situation in mapping software. MapperGUI aimed to lower the barriers to entry for those who wished to use libmapper, a software library for collaborative and configurable musical mapping. The GUI was designed to allow for quick and straightforward manipulation of musical networks. 

Techniques from data visualization and user interface design were presented to illustrate general principles used in MapperGUI's design. The ideas and structure of libmapper were summarized to describe the requirements for the GUI. Prior user interfaces for libmapper were described, as ideas and code were borrowed from them for the creation of MapperGUI.

The final GUI takes the form of a modular interface. Various independent view modes can be used interchangeably, making MapperGUI useful for a wide variety of libmapper networks. The code itself was structured in a modular fashion such that extensions could be created more easily. The program was made accessible to libmapper users throughout this project, and their feedback became a crucial factor in design decisions. 

MapperGUI has met many of the goals set out at the beginning of this thesis work. Most importantly, interface is available, functional and very accessible. In this distribution the majority of libmapper variables can be accessed and manipulated, with the notable exceptions of libmapper instances and link scopes. Within ListView, the most fully developed visual mode, connection and linking are easy and intuitive. The program presents GridView and HiveView for networks and tasks where list-type views are cumbersome, though both are not yet as fully featured as ListView. Users can save and load mappings, although these features have some notable shortcomings. The current release of MapperGUI is still in a test phase, a number of issues need to be resolved before the software can be adopted as a standard GUI for libmapper. 

\section{Future Work}
\label{sec:future_work}

	\subsection{Unimplemented features}

A few features present in Maxmapper have not yet been implemented in MapperGUI. Most importantly, MapperGUI currently does not support sending a signal as an instance. Instances are one of the true strengths of libmapper. To design a way (even an inelegant one) to allow the user to take advantage of this libmapper feature is a high priority for MapperGUI's next release. Users are also unable to edit link scopes in the current version. Support for this is in the process of being implemented via a drop-down menu on the top bar and extensions to the python monitor. 

Our search functions, though usable, are not yet quite as powerful as those found in Maxmapper. Maxmapper allows users to filter signals for common prefixes through a drop-down menu.  MapperGUI also forces users to remain a single network for each session. In the case where multiple networks are available, it would be a good extension to allow users to select and switch between them. Finally, MapperGUI's expression editing was poorly reviewed by users. When double clicking the \emph{Expression} button, Maxmapper displays a palette with all possible expression syntax (to create exponential functions, averages, etc.). Incorporating this feature would be a good start for extending the usefulness of custom expressions.

%	\begin{enumerate}
	%	\item Prefix filtering
	%	\item Network selection
	%	\item Expression Palate 
	%	\item Edit scopes
	%	\item Send as instance
%	\end{enumerate}

	\subsection{Possible extensions} % (fold)
	\label{sub:possible_extensions}

With the MVC architecture and some alternate views in place, our group has planned extensions to MapperGUI that will be interesting. Firstly, HiveView should be made fully interactive, allowing the user to create links and connections with a dragging gesture. The hierarchical edge bundling technique described in Section \ref{sec:color} would be very useful for this view, as connection lines are currently drawn somewhat arbitrarily. Attempts were made to integrate Vizmapper into MapperGUI as a single view, because it was an initial goal of the process. Unfortunately, idiosyncrasies in the Vizmapper code made this more difficult than originally anticipated. In the future we hope to restructure the Vizmapper code so that it might be included, as it is a useful network visualization. 

The research from Section \ref{sec:color} could also be applied through a new ``area'' view mode. In this mode. network elements would be displayed as shapes on a Cartesian plane with each axis being user-mappable to quantitative metadata. Other metadata could be visually mapped to these objects' colors, orientations, shapes, opacities, etc. This would be an attractive way to analyze the findings of \citeN{visual_dimensions} and may be the topic of a future project. 

The saving and loading features of MapperGUI obviously need improvement. Work has already begun on a staging process for loaded mappings, such that saved mappings are only loaded for the desired devices. 

Finally, standalone support for Windows and Linux systems is definitely required for future releases of MapperGUI. It would also be interesting to begin work on mobile versions of the software, though that would be a much more time-consuming process. 

	%\begin{enumerate}
		%\item HEB for hive
		%\item Integrate vizmapper
		%\item Save/Loading Rehaul
		%\item Standalone for Windows and Linux
		%\item Mobile support
		%\item Area view to make good on variable analysis
	%\end{enumerate}
	% subsection possible_extensions (end)


	