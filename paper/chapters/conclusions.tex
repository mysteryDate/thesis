%!TEX root = ../thesis.tex
\chapter{Conclusions \& Future Work}

This chapter summarizes the work presented this thesis, evaluates MapperGUI, and summarizes possible avenues for further research.

\section{Summary and Conclusions}

This thesis began by exploring issues relevant to musical mapping interfaces. DMI designers typically hard-code mappings into their designs, making collaboration, cross-compatibility and modification difficult. Certain tools exist to aid these designers and their users, but often they are difficult to use for non-experts in computer programming. 

This work was motivated by this situation in musical mapping software. MapperGUI aimed to lower the barriers to entry for those who wished to use libmapper, a software library for collaborative and configurable musical mapping. The GUI was designed to allow for quick and straightforward manipulation of musical networks. 

Though it has not yet been the case, in the future MapperGUI will likely be used to handle mappings in a live performance context. This will give us a new perspective on how the software performs in a situation where instant reactivity is necessary and errors can be disastrous. 

\section{Future Work}

Though MapperGUI has been released and is available for free download, there are still many possible extensions to the interface. 

	\subsection{Unimplemented features}
	\begin{enumerate}
		\item Prefix filtering
		\item Network selection
		\item Expression Palate 
		\item Edit scopes
		\item Send as instance
	\end{enumerate}
	\subsection{Possible extensions} % (fold)
	\label{sub:possible_extensions}
	\begin{enumerate}
		\item HEB for hive
		\item Integrate vizmapper
		\item Save/Loading Rehaul
		\item Standalone for Windows and Linux
		\item Mobile support
		\item Area view to make good on variable analysis
	\end{enumerate}
	% subsection possible_extensions (end)

	